\documentclass{article}
\usepackage[utf8]{inputenc}
\usepackage{xcolor}
\usepackage{hyperref}

\hypersetup{
    colorlinks=true,
    linkcolor=blue,
    filecolor=magenta,      
    urlcolor=cyan,
}
 
\urlstyle{same}




\title{Welcome-FAQ}
\author{Et al.}
\date{January 2020}

\begin{document}

\maketitle

Welcome!
This document will try to provide you with some basic information to help you get settled. It is intended to be useful rather than comprehensive. For errors and additions, ping \href{mailto:i.castro@qmul.ac.uk}{Ignacio}.



\section{London matters}
\subsection{Housing}
\textbf{Temporary accommodation}. Renting a temporary room for a couple of weeks to help bootstrap things and look for a definitive option can be quite helpful,  options include:
\begin{itemize}
    \item AirBnB/Booking
    \item Temporary university \href{https://www.universityrooms.com/en-GB/search/in/london}{accommodation}
    \item There are some cheap nearby \href{https://lhalondon.com/properties/davies-court/}{hostels}
    \item There are neighbourhood groups in facebook announcing short-term lettings, for instance in  \href{https://www.facebook.com/groups/HWSpaces/}{Hackney Wick.}
\end{itemize}

Getting accommodation is usually quite straightforward though can be intensive: you will probably need to see many places, both to get an idea and because there is many other people seeing the same place. Usually 3-5 days of relatively intensive flat visiting will be sufficient.
\begin{itemize}
    \item \textbf{Flat-sharing}: for cost reasons, a flat share is usually a good option and it has the benefit of giving you access into a new social network too. A popular site is \href{https://www.spareroom.co.uk/}{SpareRoom} 
    \item \textbf{Your own flat}: obviously a more expensive option, a popular site is \href{https://www.zoopla.co.uk//}{Zoopla}
\end{itemize}
    
\textbf{\textcolor{red}{BE AWARE}}:
\begin{itemize}
\color{red}
    \item SCAMs: do not give any money in advance, there are some scammers who might try to trick you into it.
    \item AGENCIES: housing is typically dealt by/with agencies, which typically charge fees for every contract they issue, contracts usually include a list of items present in the house and have a term but usually you can leave before the stated duration of the contract provided you find a new tenant to replace you. 
\end{itemize}

\subsection{Banking} 
Banks usually require an appointment to open an  account. There is one Santander branch and \href{http://www.hr.qmul.ac.uk/workqm/paygradingrewards/reward/benefits/index.html}{QMUL has an agreement} for staff, though proof of work will be required.
Alternatively Monzo or some other ``online banking'' option might be handy as well.

\subsection{Cycling}
If you want to use a bicycle, you can
\begin{itemize}
    \item \textbf{Rent one:} you can find ``Santander'' bikes through out most of London. There is a docking station just opposite the Engineering building. There are some other options such as ``mobike''.
    \item \textbf{Buy your own:} for second hand option check \href{https://www.gumtree.com/}{Gumtree} and/or ebay. There are  \href{http://hr.qmul.ac.uk/workqm/paygradingrewards/reward/benefits/cycletowork/}{benefits} if you want to buy a bicycle (mostly for first hand ones).
    \item There is a key-access bicycle parking within QMUL. Talk with security to be granted access to it.
\end{itemize}


\subsection{Sports and socialising}

\textbf{Sports.}

\begin{itemize}
    \item \href{https://www.qmsu.org/qmotion/}{University gym}, inside of QMUL
    \item \href{https://www.better.org.uk}{Public sport-centers}, many with  swimming-pool, gym and a range of courses/activities. Some also have sauna/jacuzzi (e.g., Mile End) and there is an all-year heated outdoor swimming-pool nearby in (i.e., London Fields lido).
    %
    With the \href{http://my.qmul.ac.uk/campus-and-city/deals-and-discounts/}{TOTUM card} (sort of a student membership card) you have full access to all facilities and courses in all the sport-centers for less than £50 a month
    \item The student union also runs different sport \href{https://www.qmsu.org/}{clubs/activities}
    \item Football: you can join matches all across London 
    \begin{itemize}
        \item on real grass: \href{https://www.meetup.com/Terrible-football-in-London/events}{``terrible football''} (no particular aptitudes required, as you can imagine), free sessions in the nearby Weabers field
        \item on artificial grass, for a \href{https://footyaddicts.com/}{fee}
    \end{itemize}
\end{itemize}      


\textbf{Social events.}
We try to organise a social event once a month. \href{mailto:n.a.arnold@qmul.ac.uk}{\href{mailto:n.a.arnold@qmul.ac.uk}{Naomi}} is in charge: ping her with ideas/proposals!

There are also many other events at \href{https://www.qmul.ac.uk/events/}{QMUL} and more generally in East-London, London's hipster heart-land


\section{QMUL-EECS}

\subsection{Where are you}
You are now in the Distributed-Computing team, within the Networks Research Group, in the School of Electronic Engineering and Computer Science, Queen Mary University of London. In the \href{http://networks.eecs.qmul.ac.uk/}{page of the group} you can find the members, seminars and announcements (e.g., grants, publications in \underline{top} conferences/journals). If you have an update let \href{mailto:g.tyson@qmul.ac.uk}{Gareth} know!  

\subsection{Bureaucracy and logistics}
\begin{itemize}
    \item \textbf{Basic information.} You can find most of the information that you will require with regards to printing, wireless/eduroam and many other things in the \href{http://support.eecs.qmul.ac.uk/}{EECS} page and the \href{https://intranet.eecs.qmul.ac.uk}{intranet}.
    
    \item \textbf{If you are a PhD student}, here is the \href{https://www.dropbox.com/s/o3tsluiwtqjbf9z/EECS-Research-Student-Handbook-2019-final-v5.pdf?dl=0}{Research Student Handbook}. If you find a more recent version, let \href{mailto:i.castro@qmul.ac.uk}{Ignacio} know.

    \item \textbf{Stationery.} Drop by CS300, ask in the reception desk (usually to \href{http://eecs.qmul.ac.uk/profiles/corkhayley.html}{Hayley Cork}). 

    \item \textbf{Keys.} Talk with the school Facilities Manager,  \href{http://eecs.qmul.ac.uk/profiles/hoskinsedward.html}{Edward Hoskin}. You will need £20 deposit for the key.
    
    \item \textbf{Printing.} Instructions for setting up printing can be found \href{http://support.eecs.qmul.ac.uk/services/staff-printing/#Mac}{here}. Our nearest release station is in Eng 1.01 (right near the men's toilets), but print jobs can be released at any EECS printer.
    
    \item \textbf{IT support. } For IT troubleshooting on things like logins, printing, eduroam, you can raise a ticket with the ITS staff \href{https://helpdesk.qmul.ac.uk/QMULServiceDesk.BridgeIT#/logon}{here}. For more systems/EECS specific issues, drop an email to the \href{mailto:eecs-systems@qmul.ac.uk}{EECS Systems team}.
\end{itemize}

\color[red]{IMPORTANT: \textbf{always close the office door} if you leave the office empty and avoid leaving valuables whenever possible: unfortunately theft is not uncommon.}

\subsection{QMUL benefits}
Students and staff can get the \href{http://my.qmul.ac.uk/campus-and-city/deals-and-discounts/}{TOTUM card} (sort of a student membership card) for discounts (e.g., 10\% at Coop,  public sport-centers flat rate).

Other benefits include the cycle-to-bike scheme. Full list  \href{http://www.hr.qmul.ac.uk/workqm/paygradingrewards/reward/benefits/index.html}{here}.

\subsection{Buddy-program}
To help new members find their way around, we run a ``buddy-program'' to partner you up with someone who has been around for longer. Ask \href{i.castro@qmul.ac.uk}{Ignacio}.


\section{Communications}

Within the group we typically rely on
\begin{itemize}
    \item office's Whatsapp group: to coordinate for lunch and other minor stuff, ask anyone to add you!
    \item We also use \href{qmul-distcomp.slack.com}{SLACK} for more general stuff
\end{itemize}    
    

There are some mail-lists you might be interested in:
\begin{itemize}
    \item \href{https://www.jiscmail.ac.uk/cgi-bin/webadmin?A0=LONDON-NETWORKING}{London-Networking:} Networking related announcements in London
    \item \href{https://www.jiscmail.ac.uk/cgi-bin/webadmin?SUBED1=NGN&A=1}{NGN}: Networking related announcements in the UK
    \item \href{mailto:LONDON-INFOSEC@jiscmail.ac.uk}{Information security} related events in London
    \item Announcements in our team, \href{mailto:distributed-computing@qmul.ac.uk}{Distributed-systems} 
    \item \href{mailto:residents@lists.eecs.qmul.ac.uk}{For all} staff, PhD students and guests working in any of the School buildings. Ends up collecting rants and almost spam. 
\end{itemize}

\subsection{Demonstrating}

Within the department there are plenty of (paid) opportunities to be involved in undergraduate/masters teaching, from assisting in labs/tutorials and supervising student projects to delivering course material. We strongly encourage all PhD students to demonstrate. There are usually two ways to register interest:
\begin{itemize}
    \item Signing up on the QMPlus termly demonstrator recruitment form, where you can select specific modules you're interested in
    \item Informally, by registering interest directly to the relevant lecturer (though you'll usually have to fill out the demonstrator interest form anyway)
\end{itemize}

EECS modules which might be particularly relevant:
\begin{itemize}
    \item Internet Protocols
    \item Computer Systems and Networks
    \item Digital Media and Social Networks
    \item 21st Century Networks
    \item Cloud Computing
    \item Applied Statistics
\end{itemize}

\subsubsection{MyHR}

\section{Academic matters}
We run two seminars:
\begin{itemize}
    \item \textbf{``Non-classical seminar'':}  intended to be  a safe space for discussion, present new ideas and get feedback. Presentations should be no longer than 30' and should clearly introduce the area (10'). You can see previous presentations and the ``manifesto'' of the seminar \href{https://drive.google.com/open?id=1miDiEmRSSHpveHaXqW6FR0U9DG8uYKeD}{here}. We run this 1h seminar twice every month and all attendants should present once, ask \href{mailto:n.a.arnold@qmul.ac.uk}{Naomi} to add you!
    \item \textbf{Seminar:} both for Networks members and externals. It is intended for presentations of ``more polished'' pieces of work. Take a look at previous \href{http://networks.eecs.qmul.ac.uk/category/seminar/}{seminars}. If you want to present or know someone who does, contact \href{mailto:d.i.ibosiola@qmul.ac.uk}{Dami} and \href{mailto:r.clegg@qmul.ac.uk}{Richard}
\end{itemize}


\end{document}
